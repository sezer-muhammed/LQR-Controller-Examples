\documentclass[11pt,twocolumn,twoside,lineno]{pnas-new}
% Use the lineno option to display guide line numbers if required.
\usepackage{noto}

\templatetype{pnasresearcharticle} % Choose template 
% {pnasresearcharticle} = Template for a two-column research article
% {pnasmathematics} %= Template for a one-column mathematics article
% {pnasinvited} %= Template for a PNAS invited submission

\title{The LQR Controller}

% Use letters for affiliations, numbers to show equal authorship (if applicable) and to indicate the corresponding author
\author[a,b]{Muhammed Sezer}
\author[a,b]{Şevval Belkıs Dikkaya} 

\affil[a]{\href{https://www.metu.edu.tr/}{Middle East Technical University}}
\affil[b]{Orion Robotics Ltd.}

% Please give the surname of the lead author for the running footer
\leadauthor{Sezer} 

% Please add here a significance statement to explain the relevance of your work
\significancestatement{This document's significance lies in its comprehensive exploration of the Linear Quadratic Regulator (LQR) controller, a key tool in the realm of system control.}

% Please include corresponding author, author contribution and author declaration information
\authorcontributions{Please provide details of author contributions here.}
\authordeclaration{Please declare any conflict of interest here.}
\equalauthors{\textsuperscript{1}A.O.(Author One) and A.T. (Author Two) contributed equally to this work (remove if not applicable).}
\correspondingauthor{\textsuperscript{2}To whom correspondence should be addressed. E-mail: author.two\@email.com}

% Keywords are not mandatory, but authors are strongly encouraged to provide them. If provided, please include two to five keywords, separated by the pipe symbol, e.g:
\keywords{Linear Quadratic Regulator $|$ LQR $|$ Control Theory $|$ System Control $|$ Practical Application $|$ Simulation $|$ Performance Analysis $|$ Case Study}


\begin{abstract}
       In this easy-to-follow guide on the Linear Quadratic Regulator (LQR) controller, you'll learn about its practical and theoretical aspects. This document takes you beyond the surprise of just how much better it is than a PID controller. It equips you with the knowledge to not only run simulations but also to apply it in real-world scenarios. From understanding the underlying principles to discussing practical applications, and even a step-by-step demonstration via a computer simulation, this guide covers it all. It also features a comparative discussion that puts LQR up against other popular controllers, so you know exactly where it stands. Whether you're a beginner or an experienced practitioner, this guide aims to give you a complete picture of the LQR controller, and how it can be a game-changer in tackling complex control problems.
\end{abstract}

\dates{\today}

\begin{document}

\maketitle
\thispagestyle{firststyle}
\ifthenelse{\boolean{shortarticle}}{\ifthenelse{\boolean{singlecolumn}}{\abscontentformatted}{\abscontent}}{}

% If your first paragraph (i.e. with the \dropcap) contains a list environment (quote, quotation, theorem, definition, enumerate, itemize...), the line after the list may have some extra indentation. If this is the case, add \parshape=0 to the end of the list environment.
\dropcap{I}magine transforming a teetering robot into a graceful performer, guiding a drone through an obstacle-strewn forest, or smoothing out a vehicle's ride with an almost magical touch. That's the LQR controller – turning challenges into walk-in-the-park realities.

Our journey will fuse theory with exciting real-world applications and immersive hands-on simulations. Picture witnessing the might of the LQR controller at work, a practical spectacle making intricate theory crystalline clear.

Get ready to dive into the riveting realm of the LQR controller, a voyage that promises not just answers but a fascinating glimpse into future possibilities in system control. Welcome aboard!

So buckle up and get ready for an exciting deep-dive into the world of the LQR controller. This guide will not only answer your questions but will also leave you marveling at the possibilities that the LQR controller opens up in the world of system control.

\section*{Introduction}
\subsection*{The Advent of the LQR Controller}

The Linear Quadratic Regulator (LQR) controller owes its birth to the era of the Space Race in the mid-20th century. As nations strove to conquer space, the need arose for control strategies capable of handling complex, nonlinear, and multi-input-multi-output systems typical of spacecraft.

In the early 1960s, engineers at NASA sought to develop a control strategy that not only provided stability but also optimally controlled the system performance. They were dealing with the challenge of guiding spacecrafts, where the task was not just to stabilize the system but also to minimize fuel consumption, a critical parameter in space travel. This marked the genesis of the optimal control theory, culminating in the development of the LQR controller.

The LQR controller provided a promising solution to these complex problems. Its underlying mathematics, based on the solution to the Algebraic Riccati Equation (ARE), allowed the control engineers to find an optimal balance between performance (such as minimizing deviation from a desired trajectory) and control effort (such as fuel consumption). The LQR controller elegantly encapsulated these two aspects into a single quadratic cost function, hence its name - Linear Quadratic Regulator.

Since its inception, the LQR controller has been applied to myriad domains beyond aerospace. It has played a significant role in fields as diverse as economics, for optimal allocation of resources; in power systems, for optimal power flow control; in robotics, for efficient and stable movement control; and in automotive systems, for enhancing ride comfort and stability, to name a few.

The beauty of the LQR controller lies in its theoretical elegance, its promise of optimal performance, and its adaptability to a wide range of control problems. This has ensured the LQR controller's continued relevance and its esteemed position in the world of control systems, from its advent during the Space Race to the present day. 

\subsection*{Purpose of the Paper}

This paper aims to serve as a comprehensive guide to the Linear Quadratic Regulator (LQR) controller. Our journey starts with exploring the theory behind this powerful tool, where we delve into its mathematical backbone, operational principles, and the inherent trade-offs involved in its design. While we will discuss abstract mathematical concepts, we strive to present them in an intuitive and accessible manner, providing readers with a concrete understanding of how the LQR controller operates.

Moreover, this paper aims to transcend the theoretical realm and venture into the practical world. Using the example of a double pendulum, a classical problem in physics that presents intriguing dynamics and challenges for control, we illustrate the application of the LQR controller. Our goal here is to demonstrate how the theoretical constructs are put into action, and how the LQR controller navigates the complex dynamics of the double pendulum.

In essence, the paper's purpose is not just to provide readers with knowledge about the LQR controller, but to equip them with the ability to understand its application in real-world scenarios. It aims to bridge the gap between theory and practice, making the understanding of the LQR controller holistic and application-oriented.

\section*{LQR Controller: A Theoretical Perspective}

\subsection*{LQR Control Principle}

At its core, the Linear Quadratic Regulator (LQR) operates on the principle of optimal control. In this context, "optimal" denotes the objective of minimizing a specific cost function. This cost function, constructed as a quadratic form, characterizes the trade-off between the state deviation from a desired reference and the control effort required to achieve that state.

To envision the principle of operation, consider the analogy of driving a car. The aim is to reach the destination (desired state) with minimum fuel consumption (control effort) and in the least time possible (state deviation). While a faster speed might help reduce the travel time, it would increase fuel consumption. Conversely, driving at a slower pace might conserve fuel but extend travel time. The LQR controller, analogous to an intelligent driving assistant, continually adjusts the speed (control input) to find an optimal balance between these conflicting objectives.

The brilliance of the LQR controller lies in its ability to make such optimal decisions at every instant, taking into account the current state of the system. This enables it to provide dynamic control, continuously adapting to changes in the system and its environment. As a result, the LQR controller stands out in scenarios where an optimal trade-off between performance and control effort is crucial.


\showmatmethods{} % Display the Materials and Methods section

\acknow{Please include your acknowledgments here, set in a single paragraph. Please do not include any acknowledgments in the Supporting Information, or anywhere else in the manuscript.}

\showacknow{} % Display the acknowledgments section

% Bibliography
\bibliography{export}


\end{document}